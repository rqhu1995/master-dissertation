\chapter{背景知识}
\label{chp:initialization}

回答集编程(Answer Set Programming,ASP)是一种声明型的编程语言,主要面向复杂的搜索问题求解。ASP程序非单调推理能力较强,因此在知识密集型应用中被广泛使用。从语法上来看,ASP程序由若干条规则组成,求解问题的推理结果不受程序中规则存放的先后顺序以及规则体中谓词存放顺序的影响,作为声明型的编程方法,ASP程序不仅表示直观、符合人们自然表达的习惯,而且可扩展性较强。从语义上来看,ASP程序以回答集语义为基础,适用于常识推理和问题建模。回答集语义的理论基础是1988年Lifschitz和Gelfond提出的经典逻辑程序的稳定模型语义(Stable Model Semantics)\cite{1992The},随后他们在该模型语义的基础上进一步扩展,推广到包含弱否定(Weak Negation)和强否定(Strong Negation)的析取逻辑程序语言中,并将稳定模型语义改名为回答集语义。在应用上来看,ASP语言简洁并且表达能力强,已经广泛应用于多种场景中,包括规划诊断\cite{2002Answer}、偏好处理\cite{2003A}、生物医学网络推理\cite{2006Modelling}等领域。

本章将分别介绍ASP程序的语法、ASP程序的语义和相关的ASP求解器。
\section{ASP程序的语法}
项(terms)是ASP中最基本的元素,项可以是变量、常量或函数。变量是以大写字母为开头的字符串(如Human、Animal)。常量是符号常量或者数字(如25、c)。函数通常包含函数符号和若干参数,其形式为$f(t_1,\ldots,t_n)$,函数中的参数也是项。若项中不存在函数符号与变量,则称该项为实例化项,否则称该项为非实例化项。

原子(atom)有谓词和项共同组成,形如$p(t_1,\ldots,t_n)$。其中,$p$为$n$元谓词的标识符,$t_1,\ldots,t_n(n \geq 0)$为项,$n$为$0$时谓词后的括号可省略。原子用于表示不同项之间的关系,例如:$bird(dove)$、$friend(tom, jerry)$、$today\_rain$,分别表示鸽子是鸟、tom和jerry是朋友以及今天下雨。如果原子中的每一项都是实例化的,则该原子是实例化原子,否则,该原子是非实例化的原子。例如,$bird(dove)$是实例化原子,$friend(X,tom)$是非实例化原子。原子$p(t_1,\ldots,t_n)$的强否定形式为$\lnot p(t_1,\ldots,t_n)$,$\lnot$是经典逻辑中的否定,$\lnot p(t_1,\ldots,t_n)$表示各项$t_i(1 \le i \le n)$不符合$p$所描述的关系,$\lnot \lnot a = a$,与经典逻辑中的排中律相同。文字(literal)可以是原子$p(t_{1},\ldots,t_n )$,也可以是原子的强否定形式$\neg p(t_{1},\ldots,t_n )$。若文字对应的原子是实例化的,则称该文字是实例化的。
\begin{definition}[规则]规则$r$是具有如下形式的式子:
    \begin{align}
        \label{rule:1}
        l_0 \lor \ldots \lor l_m \leftarrow l_{m+1}, \ldots, l_n, not\ l_{n+1}, \ldots, not\ l_p.
    \end{align}
\end{definition}
其中$p > n > m \geq 0$,$l_i$代表文字,$not$为新的逻辑连接符,通常称作缺省否定(default negation)或者失败即否定(Negation As Failure,NAF),又叫做弱否定,$not \ l_i$被读作“没有理由相信$l_i$为真”,但是这并不说明$l_i$为假。对ASP程序而言,$not\ not\ a$不与$a$等价。规则$r$读作“如果相信$l_{m+1}, \ldots, l_n$为真,并且没有理由相信$l_{n+1}, \ldots, l_p$为真,则相信$l_0\ or\ \ldots or\ l_m$为真”。

规则$r$左侧的文字是规则的头部,可以表示为$head(r)=\{l_0,\ldots,l_m\}$。规则$r$右侧的文字是规则的体部,可以表示为$body(r)=\{l_{m+1}, \ldots, l_n,not\ l_{n+1},\ldots,not \ l_p \}$,规则的体部可以细分为正体部与负体部,正体部表示为$body^+(r)=\{l_{m+1}, \ldots, l_n\}$,负体部表示为$body^-(r)=\{l_{n+1}, \ldots, l_p \}$。规则r可以表示为:
\begin{align}
    \label{rule:2}
    head(r)\leftarrow body^+(r),not\ body^-(r).
\end{align}
一个ASP程序是有限条\eqref{rule:1}所示的规则组成的集合。

根据规则的头部和体部的不同,规则具有不同的含义,具体分类如下:
\begin{enumerate}[label=(\arabic*),topsep=0pt]
    \setlength\itemsep{-0.3em}
    \item 事实(Fact):$ head(r) \neq \emptyset \ ,body(r) = \emptyset .$
    \item 约束(Constraint):$ head(r) = \emptyset \ ,body(r) \neq \emptyset .$
\end{enumerate}

根据公式\eqref{rule:1},规则可以分为正规规则、缺省规则和严格规则。
\begin{definition}[正规规则]当规则r中$|head(r)|=1$时,规则被称为正规规则。
\end{definition}
\begin{definition}[缺省规则]当规则r中$body(r)^- \neq \emptyset$时,规则被称为缺省规则。
\end{definition}
\begin{definition}[严格规则]当规则r中$body(r)^- = \emptyset$时,规则被称为严格规则。
\end{definition}
除了\eqref{rule:1}所示的规则表现形式外,还出现了选择规则(choice rule),它的表示形式如下:
\begin{align}
    \label{rule:3}
    a \{p(X):q(X)\} b \leftarrow body(r) .
\end{align}
式子中的$a$,$b$是非负整数,可以省略不写。选择规则表示,当规则的体部满足时,可以得到集合$S$,且$a \leq |S| \leq b$,S中的元素形如$p(t)$。


\begin{definition}[外部函数]使用脚本语言LUA或者Python时,gringo可以通过调用外部的LUA、Python的函数来丰富输入的语言,格式是在调用的函数前加@号,在本文中,本文假定所有的外部函数都具有确定性,即当一个函数在相同的参数下被调用多次时,他会返回相同的值;如果外部函数在执行时发生错误,程序会打印警告并删除当前实例的规则或条件。
\begin{example}
假设$distance(x_1,y_1,x_2,y_2)$函数是python函数,功能是求两个坐标之间的欧几里得距离,该函数的实现形式如下:

\begin{tcolorbox}
\setmonofont{Source Code Pro} % I don't have Consolas
\begin{minted}{python}
import clingo
def distance(x_1, y_1, x_2, y_2):
    x = (x_1 - x_2) ** 2
    y = (y_1 - y_2) ** 2
    ans = (x + y) ** 0.5
    return ans
\end{minted}
\end{tcolorbox}

该函数可以用于以下回答集程序:

$ r_1: \ s(2,5).$

$ r_2: \ t(5,9)$

$ r_3: \ ok(X_1,Y_1,X_2,Y_2) \leftarrow \  @distance(X_1,Y_1,X_2,Y_2)\ < \ 10\ , \ s(X_1,Y_1), \ t(X_2,Y_2).$

上述例子中调用了外部函数$distance$计算$s$与$t$之间的欧几里得距离,经过$distance$计算后得到距离为5,小于10,因此满足规则$r_3$。
\end{example}
\end{definition}
\section{ASP程序的语义}
\subsection{程序的解释和可满足性}
\begin{definition}[基础逻辑程序]基础逻辑程序$P$是若干规则组成的集合,如下所示:
    $P \ = \ \{r_1,r_2,...,r_n \ | \ n \ > \ 0\}$
    在基础逻辑程序的研究领域内,不仅涉及程序、规则和文字,还涉及$Herbrand$基、$Herbrand$域和程序的模型与解释等概念。因此,本文也就依次介绍这些概念。
\end{definition}
\begin{definition}[Herbrand域H]使用$P$代表逻辑程序,将程序$P$中出现的常量和函数形成的所有基项的集合称为$Herbrand$域,使用$B_P$表示。
\end{definition}
\begin{definition}[Herbrand基]将$Herbrand$域内的全部原子和程序$P$所出现的谓词符号组成的集合称为$Herbrand$基,使用$A_P$表示,为了保持简洁,通常用$A$来代替$A_P$。
\end{definition}
\begin{definition}给定一个规则$r$,使用$ground(r)$表示所有规则的集合,这些规则是通过使用常量持续地替换$r$中的变量而获得的。
\end{definition}
\begin{definition}一个程序$P$是一系列规则的集合,具有变量的程序被理解为$P$中规则的所有基本实例的集合的简写,表示如下:
    \begin{align}
        \label{rule:4}
        ground(P) \ = \ \bigcup_{r \in P} \ ground(r)
    \end{align}
\end{definition}

一个解释$I$由一对$<I^+,I^->$组成,并且$ I^+ \cup I^- \subseteq A$,$I^+ \cap I^- = \emptyset$。$I^+$收集已知为真的原子知识,$I^-$收集已知为假的原子知识。当$I^+ \cap I^- = \ A$时,$I$被称作完全解释,如果$I$为不完全解释,这代表着解释$I$中存在真值未确定的原子。

假设$P$是一个回答集程序,$I$时一个解释。当一个正文字$a$满足$a \in I^+$时,解释$I$满足$a$,记作$ I \models a$,当一个NAF文字$not \ a$满足$a \in I^-$时,解释$I$满足$not \ a$,记作$ I \models not \ a$。当解释$I$满足一个集合$S$的每一个文字时,解释$I$满足集合$S$,记作$ I \models S$,当一个规则$r$满足$ I \nvDash body(r) \ or \ I \models head(r)$时,解释$I$满足规则$r$。当解释$I$满足程序P的所有规则时,解释$I$是程序$P$的一个模型。假设$a$是程序$P$的一个原子,当$head(r)=a \ and \ I \models body(r)$时,解释$I$支持$a$。


\begin{definition}[程序的模型]假设$P$是一个程序,$ground(P)$是程序$P$中所有基实例的集合,程序$P$的一个解释是$P$的一个模型,在该解释下,满足$ \forall a \ \in  \ ground(P)$为真。
\end{definition}
\subsection{回答集语义}
\begin{definition}[一致性文字集合]给定一个集合$S$,加入$S$中不同时包含$a$和$\neg a$,则集合$S$是一致性文字集合,其中,$a$是任意文字。
\end{definition}


\begin{definition}[可满足性,Satisfiability]给定集合$S$,文字$lit$满足集合S是指文字$lit$属于集合$S$,即$lit \in S$。集合$S$满足头部$head(r)$是指$head(r)$与S的交集不为空集,即$S \cap head(r) \neq \emptyset$。集合$S$满足体部是正体部包含于集合$S$,并且负体部和集合$S$的交集是空集,即 $body^+(r) \subseteq S$,$S \cap body^-(r) = \emptyset$。集合$S$满足规则$r$是指$S$满足规则$r$的头部或者$S$不满足规则$r$的体部。如果一个集合$S$满足规则$r$,记作$S \models r$,给定一个ASP程序$P$,如果对于$P$中的任意规则r,$S\models r$,那么就说$S$满足程序$P$。
\end{definition}
\begin{example}
给定ASP程序$P$,包含两条规则:

$r1: p \leftarrow q,\ s. $

$r2:s \leftarrow t.$

以及一个实例化文字集合$S$为:

$\{ p, \ q \}$

根据上述定义检查定义检查集合$S$是否满足程序$P$。
\begin{enumerate}[label=(\arabic*),topsep=0pt]
    \setlength\itemsep{-0.3em}
    \item 集合$S$满足文字$p$与文字$q$;
    \item 因为$S \cap head(r1) = p$,所以集合$S$满足$head(r_1)$,集合$S$满足规则$r_1$;
    \item 因为$ body^+(r2)$不含于集合$S$,所以集合$S$不满足$body(r_2)$,因此集合$S$满足规则$r_2$;
    \item 因为集合$S$满足程序$P$的每一条规则,因此,集合$S$满足程序$P$。
\end{enumerate}
\end{example}
\begin{definition}[规约,Gelfond-Lifshitz]假设程序$P$是给定的实例化程序,$S$是集合,$l$是文字,$S \subseteq Lit(P), \ l \in Lit(P)$,$P$关于$S$的GL规约结果是$P^S$,根据以下过程求得:
    \begin{enumerate}[label=(\arabic*),topsep=0pt]
        \setlength\itemsep{-0.3em}
        \item 如果$l \in S$,将程序$P$中所有包含$not \ l$的规则删除;
        \item 删除剩下规则中的缺省文字;
    \end{enumerate}
\end{definition}
ASP的回答集语义包含两种情况,第一种情况考虑不包含缺省否定的程序,即简单逻辑程序,另一种情况是考虑包含缺省否定的程序,即扩展逻辑程序,下面分别介绍这两种情况。
\begin{definition}[简单逻辑程序回答集]假设$P$是不包含缺省否定的程序,程序$P$的回答集,记作$M(P)$,该集合$S$满足:
    \begin{enumerate}[label=(\arabic*),topsep=0pt]
        \setlength\itemsep{-0.3em}
        \item 对于程序$P$中的每一条规则$l_0 \leftarrow \ l_1,...,l_m$,如果$l_1,...l_m \ \in \ S$,那么$l_0 \ \in \ S$。
        \item $S \ \subseteq \ lit(P)$且不存在$S$的任何子集也满足程序$P$的每一个规则,即$S$是满足程序$P$的最小集合;
        \item 如果$S$包含含互补的文字,那么$S \ = \ lit(P)$;
    \end{enumerate}
\end{definition}
\begin{definition}[扩展逻辑程序回答集]假设P是包含缺省否定的程序,$S$是实例化文字集合,经过GL规约后,如果$S$是$P^S$的回答集,则$S$是$P$的回答集。如果程序$P$没有回答集或者只有一个回答集$Lit(P)$,那么程序$P$是不一致的,否则,程序$P$是一致的。可以看出,使用GL规约,可以将扩展逻辑程序程序转换为简单逻辑程序,从而得到扩展逻辑程序的回答集。下面通过一个例子说明
\end{definition}

\begin{example}

给定实例化后的ASP程序$P$:

$r_1: \ p(a) \leftarrow not \ p(b).$

$r_2: \ p(b) \leftarrow not \ p(a).$

$r_3: \  \leftarrow p(b).$

这段程序可能的解释有四个: $S_1 = \emptyset , \ S_2  =\{p(a)\}, \ S_3  =\{p(b)\},S_4  =\{p(a),p(b)\}$。依次验证:
\begin{enumerate}[label=(\arabic*),topsep=0pt]
    \setlength\itemsep{-0.3em}
    \item 当$S_1 = \emptyset$时,由于$S_1$中不存在文字,因此不满足GL规约的条件1,删除规则中的缺省文字,此时,$P^S_1 = \{r_1: \ p(a). \  r_2: \ p(b).\ r_3: \  \leftarrow p(b).\}$,由于集合中的规则$r_2$与$r_3$存在矛盾,因此没有回答集,$S_1$不是P的回答集。
    \item 当$S_2 = \{p(a)\}$时,根据GL规约,首先删除包含$not \ p(a)$的规则$r_2$,再删除掉规则$r_1$中的缺省文字$not \ p(b)$,$P^S_2 = \{r_1: \ p(a). \ r_3: \  \leftarrow p(b).\}$,所以$P^S_2$的回答集$S = \{p(a)\} \ = S_2$,所以$S_2$是程序P的回答集。
    \item 当$S_3 = \{p(b)\}$时,根据GL规约,首先删除包含$not \ p(b)$的规则$r_1$,再删除掉规则$r_2$中的缺省文字$not \ p(a)$,$P^S_3 = \{r_2: \ p(b). \ r_3: \  \leftarrow p(b).\}$,由于集合中的规则$r_2$与$r_3$存在矛盾,因此没有回答集,$S_3$不是P的回答集。
    \item 当$S_4 = \{p(a),\ p(b)\}$时,根据GL规约,首先删除包含$not \ p(a)$的规则$r_2$,包含$not \ p(b)$的规则$r_1$,$P^S_4 = \{r_1: \ p(a). \ ,r_2: \ p(b). \ r_3: \  \leftarrow p(b).\}$,由于集合中的规则$r_2$与$r_3$存在矛盾,因此没有回答集,$S_4$不是P的回答集。
\end{enumerate}
\end{example}
\begin{definition}
回答集的语义高度依赖于出现在程序负文字中原子的正确性,为了方便使用,本文使用$NANT(P)$来表示出现在NAF文字中的原子,公式定义如下:
\begin{align}
    \label{rule:5}
    NANT(P) \ = \ \{a \ | \ \exists \ r \ \in ground(P)\ : \ a \ \in \ body^-(r) \}
\end{align}
\end{definition}
\begin{definition}
通常来说,一个程序$P$由多个不同的回答集,本文定义$SM(P)$,来表示程序$P$所有回答集的集合。
\end{definition}
\subsection{well-founded 语义}
上世纪90年代,良基语义的概念由Van Gelder A提出,对于任意的回答集程序,都存在对应的良基模型,并且在多项式时间内\cite{van1991well}可以计算出这个模型。
\begin{definition}可以假设$P$是回答集程序,$S$和$V$是$A$中原子的集合,可以将集合$S$关于$P$和$V$的立即结果$T_{P,V}(S)$定义如下:
    \begin{align}
        \label{rule:6}
        T_{P,V}(S) \ = \ \{a| \ \exists r \in \ P,head(r) \ = \ a,body^+(r) \subseteq S, body^-(r) \cap V = \emptyset \ \}
    \end{align}
    根据上式,可以得到,如果集合$V$确定了,关于集合$S$的函数$T_{P,V}(S)$是递增的。因此,可以定义函数$lfp(.)$来表示当集合$V$固定时,函数的最小定界点。
\end{definition}
\begin{definition}假设$P$是程序,$P^+$是程序$P$中不包含负体部的规则集合,序列$(K_i,U_i)_{i \geqslant 0}$定义如下:

$K_0 \ = \ lfp(T_{P^+})$

$K_i \ = \ lfp(T_{P,U_{i-1}})$

$U_0 \ = \ lfp(T_{P,K_0})$

$U_i \ = \ lfp(T_{P,K_i})$

假设$j$是计算过程中的第一个下标,则计算方法形如$<K_j,U_j> \ = \ <K_{j+1},U_{j+1}>$,通常使用$WF_P \ = \ <W^+,W^->$来表示程序$P$的唯一良基模型,其中,$W^+ \ = \ K_j$并且$W^- \ = \ A\setminus U_j$。
\end{definition}
\begin{example}
假设程序$P$包含以下四条规则:

$r_1: \ q \ \leftarrow \ a,not \ p.$

$r_2: \ p \ \leftarrow \ a,not \ q.$

$r_3: \ a \ \leftarrow \ b.$

$r_1: \ b.$

该程序$P$的$K_0$的计算过程如下:
\begin{enumerate}[label=(\arabic*),topsep=0pt]
    \setlength\itemsep{-0.3em}
    \item 首先计算$P^+$,$P^+ \ = \ \{\ a \ \leftarrow \ b,  \ b.\}$。
    \item 当$S \ =\ \emptyset$时、,因为规则$r_4$满足$body^+(r_4) \ \subseteq\ S, \  body^-(r_4) \ \cap\ V = \ \emptyset$,因此,$T_{P^+}(\emptyset) \ = \ \{b\}$。
    \item 当$S \ = \ \{a\}$时,因为规则$r_3,r_4$的正体部体部不含于$S$,所以规则$r_3,r_4$都不满足条件,因此,$T_{P^+}(\{a\}) \ = \ \emptyset$。
    \item 当$S \ = \ \{b\}$时,因为规则$r_3,r_4$都满足条件,即正体部含于集合$S$,负体部交集合$V$为空,因此,$T_{P^+}(\{b\}) \ = \ \{a,b\}$。
    \item 当$S \ =\ \{a,b\}$时,因为规则$r_3,r_4$都满足条件,即正体部含于集合$S$,负体部交集合$V$为空,因此,$T_{P^+}(\{a,b\}) \ = \ \{a,b\}$。
    \item 因此,$T_{P^+} \ = \ \{a,b\}, \ K_0 \ = \ lfp(T_{P^+}) = \ \{a,b\}$。
\end{enumerate}
该程序$P$的$U_0$的计算过程如下:
\begin{enumerate}[label=(\arabic*),topsep=0pt]
    \setlength\itemsep{-0.3em}
    \item 此时,集合$V \ = \ K_0 =  \{a,b\}$。
    \item 当$S = \emptyset$时,因为规则$r_4$满足$body^+(r_4) \subseteq S, \  body^-(r_4) \cap V = \emptyset$,因此,$T_{P^,K_0}(\emptyset) \ = \ \{b\}$。
    \item 当$S = \{a\}$时,因为规则$r_1,r_4$都不满足条件,规则$r_2,r_3$满足条件,因此,$T_{P^,K_0}(\{a\}) \ = \ \{b,q\}$。
    \item 当$S = \{b\}$时,因为规则$r_3,r_4$都满足条件,规则$r_1,r_2$不满足条件,因此,$T_{P^,K_0}(\{b\}) \ = \ \{a,b\}$。
    \item 当$S = \{a,b\}$时,因为规则$r_1,r_2,r_3,r_4$都满足条件,因此,$T_{P^,K_0}(\{a,b\}) \ = \ \{a,b,p,q\}$。
    \item 因此,$T_{P^,K_0} \ = \ \{a,b,p,q\} \ U_0 \ = \ lfp(T_{P^,K_0}) = \ \{a,b,p,q\}$。
\end{enumerate}
同理,可以得出$K_1 \  = \ \{a,b\} \ = \ K_0, \ U_1 \ = \{a,b,p,q\} \ = \ U_0$,因此,程序$P$的良基模型为$<\{a,b\},\emptyset>$,$p$和$q$在良基模型中属于undefined。
\end{example}



\section{ASP求解器}
一般来说,ASP程序的求解器包括两层体系结构,第一层是实例化工具(Grounders),实例化工具把ASP程序中的变量实例化为常量,并实现选择规则、内置谓词等扩展功能;第二层是模型搜索引擎,这是ASP求解器的核心部件,主要功能是处理经过实例化所得到的ASP程序,并对搜索过程进行优化处理,高效的计算出回答集。

迄今为止,已经出现了许多高效的ASP求解器,例如Smodels\cite{2000Proceedings}、DLV\cite{2005The}、Cmodels\cite{2004SAT}、ASSAT\cite{lin2004assat}、CLASP\cite{gebserGringoNewGrounder2007}和Clingo\cite{2007clasp}等,这其中,使用最为广泛的ASP求解器是由P.Simons、I.Niemela等人开发的Smodels/Lparse。Smodels/Lparse系统支持部分额外类型的文字,例如基数文字,以$L\{l_1,...l_n\}U$的形式存在,$L$和$U$是正整数,并且$L \ \leqslant \ U$,$l_1,...,l_n$都是文字,如果$\{l_1,...,l_n\}$中有$x$个文字是正确的并且$L \ \leqslant x \ \leqslant U$,则说明集合$M$满足基数文字$\{l_1,...,l_n\}$。

Lparse作为前端输入框架,Lparse的输入为一个逻辑程序$P$,Lparse的输出为$ground(P)$的简洁输出。Smodels以Lparse的输出作为输入,并且调用相关程序来得到逻辑程序$P$的回答集。Smodels作为后端引擎,产生输入程序的答案集集合,可以提供各种控制选项来指导计算。例如:可以对提供的回答集数量建立限制,或者要求答案集包含特定的原子等。

除了Smodels求解器外,常见的ASP求解器还包括:N.Leone、W.Faber和G.Pfeifer等人开发的DLV求解器;R.Laminski、M.Gebser、B.Kaufmann等人开发Clingo求解器。DLV求解器以良基(well-founded)模型语义为基础,结合Herbrand模型,提出了析取逻辑程序回答集的递归算子,并配合回答集的检验算法,来完成回答集的计算;借鉴了SAT问题的求解算法DPLL,Clingo求解器定义了冲突集,改进了冲突驱动子句学习技术和冲突驱动回溯策略,提出了基于可满足性检测SAT和约束处理CSP的方法来计算程序的回答集。DLV求解器的前端需要完成模型建立、智能实例化和模型检查三项任务,花费时间较长,为了减少运算时间,DLV在内核采用多种优化技术(例如:模型建立优化、实例化优化等)。Clingo求解器的前端需要将程序实例化,通常使用Gringo来完成这项任务,Clingo求解器的后端Clasp用于回答集的求解。Clingo将Gringo和Clasp进行结合,并在证明回答集是否有解进行了优化。Clingo求解器与DLV求解器因为具有求解能力出色、开源和免费的特点,已经投入到许多商业化模式中,并在不断完善求解能力和效率。

可以注意到现在常用的ASP求解器,它们的运作方式都类似于Smodels。DLV使用自己的$grounder$,其他的求解器使用Lparse。后续新提出了一些$grounder$程序,例如Gringo、SAT等。

因为Clingo求解器具有免费开源、使用简单、运行高效的特点,本文在求解ASP程序时,使用Clingo进行相关实验在求解well-founded模型时,采用DLV求解器。


