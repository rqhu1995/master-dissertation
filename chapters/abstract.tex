%% ----------------------------------------------------------------------------
%%                              Chinese Abstract
%% ----------------------------------------------------------------------------

\begin{abstract}{回答集程序,逻辑程序图表示,回答集程序解释发现,回答集程序调试,人工智能可解释}
    随着人工智能技术的发展,利用人工智能进行辅助决策的方法变得越来越复杂,导致求解方案的可解释性下降。因此,对人工智能可解释的需求越发重要。回答集编程(Answer Set Programming,ASP)作为人工智能领域的编程范式之一,由于其强大的表达力,被广泛应用于各领域问题的求解。然而,ASP程序的推理求解对使用者而言是一个“黑盒” ,用户在求解过程中无从得知程序的结论是如何推导得到的;站在编程人员的角度,面对非预期的求解结果,只能采用逐个文字、逐条规则的方式进行程序错误的排查,效率低下。鉴于此,设计实现一个ASP程序的解释与调试系统,对促进ASP在各领域辅助决策的成功运用至关重要。现有的ASP程序解释与调试工具难以在结果的可读性与内容的完整性之间达到平衡,当程序规模较大或较复杂时,多面临结果过于复杂,可读性较差或结果过于简洁,解释不完整的问题。为了解决这些问题,本文考虑设计可交互的ASP程序解释与调试模型,通过交互将用户的偏好引入解释与调试的过程中,以达到可读性与完整性之间的平衡,本文的主要工作如下:
    \begin{enumerate}[label=\arabic*.,topsep=0pt]
        \setlength\itemsep{-0.3em}
        \item 参考ASP程序求解解释的现有工作,设计一致性ASP程序的解释模型,包括对一致性程序关于某文字解释的定义、解释过程中的用户交互机制的研究、用户交互下的非实例化ASP程序的实例化方案的构建,最后提出一致性ASP程序的解释生成算法;
        \item 在解释模型的基础上,结合不一致ASP程序的已有理论研究,进一步设计不一致ASP程序的调试模型。包括不一致ASP程序调试入口的推荐、单步调试的定义,在不一致程序无法获得回答集的情况下,提出一种用户交互下基于文字-规则依赖图标记的ASP程序调试算法;
        \item 基于ASP程序解释与调试模型,设计并实现ASP程序解释和调试系统,通过对一致性ASP程序进行求解结果解释,对不一致ASP程序进行调试,验证本文所设计的两个模型的有效性。
    \end{enumerate}

\end{abstract}

%% ----------------------------------------------------------------------------
%%                              English Abstract
%% ----------------------------------------------------------------------------
\begin{englishabstract}{Answer Set Programming, Explanation Finding for Logic Programs, Debugging for Logic Programs}
This article proposes a new Southeast University master degree thesis
\end{englishabstract}
